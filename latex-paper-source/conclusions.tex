% !TEX root = template.tex

\section{Concluding Remarks}
\label{sec:conclusions}
In this paper I explored the performance of various deep learning approaches based on CNNs for HAR using a single wearable sensor. The goal was to improve the result of \cite{base-paper} and prove that automatic feature extration would result in high classification accuracy. This task was successfully tackled using a 8-layer CNN and the proposed data augmentation methods. The combination of rotational and permutational data augmentation methods improves the baseline \mbox{F1-Score} of 0.961 to 0.967, overall improvement is not substancial but specific underrepresented classes received most benefit.
By comparing CNNs with 1D (temporal-only) convolutions, CNNs with both 1D and 2D convolutions, ResNets and stacked denoising autoencoders, the best model was a 8-layer CNN with both 1D and 2D convolutions capturing also the cross-correlation between different sensor values. I managed to achieve an \mbox{F1-Score} of 0.972 on the best performing model.
Many recent research papers as \cite{nils-2016, Valarezo-2017, su-2016} show that recurrent neural networks consistenly outperform deep neural networks and CNNs in HAR tasks and this would definitely be an interesting architecture to test against the ones developed in this paper. I did not experiment with different sampling rates, which is a crucial topic when dealing with wearable devices: reduced sampling rates imply more efficient resource use in \mbox{real-world} deployments. Minimal changes in the model architecture may result in noticeable improvement in the classification accuracy, as I showed in \secref{sec:final_results}. Random exploration of the parameter space for each model would be an effective strategy to iteratively \mbox{fine-tune} selected models as shown by \cite{nils-2016}.
Moreover, aside from the technical challenges I had to overcome to develop this project, I realised the importance of reading and undestanding research material on the topic of interest, specifically machine learning applied to HAR tasks. Coming up with effective models requires both a deep understanding of machine learning and a background knowledge of HAR. Recent research papers are a great starting point to be aware of the state of the art architectures for the topic one's working on. This awareness gave me the motivation to write a clear, concise and understandable research paper and to open source my work at https://github.com/damnko/har-convolutional-neural-networks so that hopefully my effort can help the community.
As a last thing, this was my first opportunity to realise the importance of being confident in working with datasets and coming up with creative solutions to rearrange and elaborate data. The data preprocessing pipeline is as important as developing and testing the model and may require a substancial investment of time.